\documentclass[PMO,lsstdraft,authoryear,toc]{lsstdoc}
% lsstdoc documentation: https://lsst-texmf.lsst.io/lsstdoc.html
\input{meta}

% Package imports go here.

% Local commands go here.

%If you want glossaries
%\input{aglossary.tex}
%\makeglossaries

\title{Network Infrastructure as a Code}

% Optional subtitleIn
% \setDocSubtitle{A subtitle}

\author{%
Hernán Stockebrand
}

\setDocRef{ITTN-060}
\setDocUpstreamLocation{\url{https://github.com/lsst-it/ittn-060}}

\date{\vcsDate}

% Optional: name of the document's curator
% \setDocCurator{The Curator of this Document}

\setDocAbstract{%
In this document we will explain the benefits and improvements of the network automation by creating peace of codes for iterative and complex management task
}






% Change history defined here.
% Order: oldest first.
% Fields: VERSION, DATE, DESCRIPTION, OWNER NAME.
% See LPM-51 for version number policy.
\setDocChangeRecord{%
  \addtohist{1}{YYYY-MM-DD}{Unreleased.}{Hernán Stockebrand}
}


\begin{document}



% Create the title page.
\maketitle
% Frequently for a technote we do not want a title page  uncomment this to remove the title page and changelog.
% use \mkshorttitle to remove the extra pages

% ADD CONTENT HERE
% You can also use the \input command to include several content files.
\section{Introduction}

When organization automate some of their iterative network task like a backup for example deploy proprietary software that requires extensive training.

Manually managing long and complex methods for procedures results in a delay of the team agility, So Network teams are becoming away from the DevOps Movement, which power up network task.

\section{Network Automation}

Network Automation is the process of automating the changes of configuration, upgrade, backup and maintenance and anothers tasks of Networks and related services. 

Network automation, convert tasks and procedures done at each stage of the network management into automated tasks that can complete them consistently and reliably with control version changes.





\appendix
% Include all the relevant bib files.
% https://lsst-texmf.lsst.io/lsstdoc.html#bibliographies
\section{References} \label{sec:bib}
\renewcommand{\refname}{} % Suppress default Bibliography section
\bibliography{local,lsst,lsst-dm,refs_ads,refs,books}

% Make sure lsst-texmf/bin/generateAcronyms.py is in your path
\section{Acronyms} \label{sec:acronyms}
\addtocounter{table}{-1}
\begin{longtable}{p{0.145\textwidth}p{0.8\textwidth}}\hline
\textbf{Acronym} & \textbf{Description}  \\\hline

PMO & Project Management Office \\\hline
VPN & virtual private network \\\hline
\end{longtable}

% If you want glossary uncomment below -- comment out the two lines above
%\printglossaries





\end{document}
